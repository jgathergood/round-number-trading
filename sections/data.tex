\section{Data}

Data were provided by Barclays Stockbroking, an execution-online brokerage service operating in the United Kingdom. The data cover the period April 2012 to March 2016 and include daily-level records of trades and quarterly-level records of portfolio positions.\footnote{During the data period the brokerage operated only through an online interface. Barclays have subsequently introduced a mobile phone trading app.} The data also include a dummy variable, at daily frequency, denoting whether the investor made a login to their account on the day. The daily-level login dummy variable covers all days, including days on which the market is closed such as Sundays and public holidays, which we use later in our analysis. We combine the daily-level records of trades with the quarterly-level records of portfolio positions, together with stock price data from Datastream, to calculate the value of each stock position in an investor's portfolio on each day of the sample period.

\subsection{Sample Selection}

As a first step, we apply a series of data cleaning sample restrictions which restrict the data to active accounts with trading histories during the data period for which we can match price and demographic data. Details of this first stage of data cleaning are shown in \ref{tab:sample_selection}. The unrestricted sample as received from Barclays contains 155,300 accounts. In this version of the paper we draw a 60\% random sample of accounts for analysis. 

The unit of observation in the data is an account $\times$ stock $\times$ day, i.e. an observation per investor per stock holding per day. We focus our analysis on two subsets of this universe of account $\times$ days, specifically login-days and sell-days. We define a login-day as an observation which is paired with a login and a sell-day as an observation which is paired with a sale event on the day from the portfolio (of the stock, or of a different stock held in the account on the same day). The sample of accounts together provides a total of approximately 67 million login-days and 500,000 sell-days.

We then apply five data cleaning restrictions, which are applied to the data at the account level unless otherwise noted. We apply these restrictions in order to limit the sample to the minimum variables required for analysis. First, we drop observations for which the account is inactive, defined as a one-year period in which the investor makes fewer than two logins or two transactions. Where an account does not meet this restriction, we drop all observations for the relevant year.\footnote{In cases where the account satisfies this sample restriction in other years, we keep those years of observations in the data set.} Second, we remove observations where a matched price is not available from Datastream. Third, we remove observations for all account $\times$ days in which there are fewer than two stocks within the portfolio.  Fourth, we remove all observations for accounts for which demographic data is missing (i.e., we drop all investor $\times$ stock $\times$ days for that account from the sample). Finally, we remove the days in which the investor purchased the stocks (starting position days) as speculative day traiding is rare among retail investors. 

\ref{tab:sample_selection} reports the effects of these steps in sample selection. The table reports the number of accounts dropped due to each step in the sample restrictions, together with the number of login events and sell events (account $\times$ stock $\times$ days) dropped at each step. From the starting sample of approximately 46,000 accounts, the largest drop of accounts is due to dropping approximately 14,400 inactive accounts (31.3\% of accounts). After applying all five sample restrictions the resulting baseline sample retains 58.8\% of accounts from the unrestricted sample. Our sample restrictions tend to drop accounts with below-average logins and sales (due to the largest drop being the drop of inactive accounts), hence the baseline sample retains 64.8\% of login-days and 70.9\% of sell-days. 

As a second step, we restrict to a sample for analysis. Two motivations drive our sample selection. First, responses to changes in left-digits are only detectable in a sample of observations for an investor in which the left-digit changes.  A key element in our analysis therefore is to draw a ``price increasing sample'' and a ``price decreasing sample'', which we define below. Moreover, we show that the response of changes in left-digit is very different depending upon when the stock is increasing in value or decreasing in value over time, in particular, selling activity occurs when prices cross left-digits from below and from above.

Second, responses to changes in left-digits are contingent upon the investor observing the change in left-digit. For example, a stock that changes left-digit over a holding period in which the investor does not make a login to the account is much less likely to be noticed compared with a change in left-digit which occurs between login days within a holding period. We therefore apply sample restrictions in order to obtain a series of observations in which the price crosses the left-digit between login-days.

We define the price increasing sample and the price decreasing sample as follows. First, using the example of the price increasing sample, we identify the first day in each calendar quarter on which an investor made a login to their account.\footnote{We show later that results are unchanged when we modify the period that defines a sample to either a month, or a year, instead of a quarter.} We then define the price increasing sample as the set of login days within the quarter for which the prices on subsequent login days were always above the price on the first day and the left-digit had changed within the quarter on at least one subsequent login-day. This sample therefore provides a series of login-days through the quarter in which the price of the stock had broached a left-digit change on at least one of the login-days. 

We define the price decreasing sample using parallel sample restrictions applied to decreasing prices. Hence the price decreasing sample is defined as the set of login days within the quarter for which the prices on subsequent login days were always below the price on the first day and the left-digit had changed within the quarter on at least one subsequent login-day. Our samples are based on quarters and individual $\times$  login days during the quarter. 

\subsection{Summary Statistics}

\ref{tab:summary_stats} provides summary statistics for account holder characteristics and account characteristics in the baseline sample. Approximately 80\% of account holders are made and the average age of an account holder is 55 years. Account holders have held their accounts with Barclays for, on average, approximately five years, with approximately 25\% of account holders having held their account for over seven years. The average portfolio value is approximately ..., with accounts containing of average ... stocks. Investors in the baseline sample overwhelmingly hold positions in a few common stocks. Holdings of mutual funds account for only 8\% of the average investor's portfolio. Investors in the sample login approximately once per five days, but trade much less frequently at a frequency of approximately once every thirty days.

\ref{tab:price_summary_stats_main} describes the price data for the baseline sample, price increasing sample and price decreasing sample. The baseline sample provides approximately 43.9 million login-day observations (the bottom row of \ref{tab:sample_selection}). Panel A summarises prices of all observations paired with login-days and sell-days in the first two rows, together with price of stocks sold in the third row. The mean price of a stock in the sample of login-days is approximately \pounds8, with a median of \pounds3. 

Panels B and C summarise prices for stocks from observations in the price increasing sample and observations in the price decreasing sample. Note, there are four units of left-digit in the data, pennies, tens of pennies, pounds and tens of pounds (there are only a few cases of hundreds of pounds). So, the left-digit changes of interest are pence to tens of pence, tens of pence to pounds, and pounds to tens of pounds (plus a few cases of tens of pounds to hundreds of pounds). The most common price range for observations in both the price increasing sample and the price decreasing sample is the \pounds1.1 to \pounds10.1 range, which accounts for 54.8\% of observations in the price increasing sample and 43.4\% of observations in the price decreasing sample. \footnote{Most stocks in the samples are prices in the range \pounds1.10 to \pounds10.10. A histogram of prices for all investor $\times$ login days is shown in \ref{fig:histogram_prices}.}


