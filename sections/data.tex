\section{Data}

Data were provided by Barclays Stockbroking, an execution-online brokerage service operating in the United Kingdom. The data cover the period April 2012 to March 2016 and include daily-level records of trades and quarterly-level records of portfolio positions.\footnote{During the data period the brokerage operated only through an online interface. Barclays have subsequently introduced a mobile phone trading app.} The data also include a dummy variable, at daily frequency, denoting whether the investor made a login to their account on the day. The daily-level login dummy variable covers all days, including days on which the market is closed such as Sundays and public holidays, which we use later in our analysis. We combine the daily-level records of trades with the quarterly-level records of portfolio positions, together with stock price data from Datastream, to calculate the value of each stock position in an investor's portfolio on each day of the sample period.

\subsection{Sample Selection}

As a first step, we apply a series of data cleaning sample restrictions which restrict the data to active accounts with trading histories during the data period for which we can match price and demographic data. Details of this first stage of data cleaning are shown in \ref{tab:sample_selection}. The unrestricted sample as received from Barclays contains 45,919 accounts. The unit of observation in the data is an investor $\times$ stock $\times$ day, i.e. an observation per investor per stock holding per day. We focus our analysis on three subsets of this universe of investor $\times$ stock $\times$ days, specifically login-days, transaction-days and sell-days. We define a login-day as a day on which the investor made a login to the account, a transaction day as a day with at least one transaction and a sell-day as a day with at least one sale. The sample of accounts together provides a total of approximately 67 million login-days, 1.2 million transaction-days and 500,000 sell-days.

We then apply five data restrictions, which are applied to the data at the account level unless otherwise noted. First, we drop observations for which the account is inactive, defined as a one-year period in which the investor makes fewer than two logins or two transactions. In such cases, we drop all investor $\times$ stock $\times$ days from the sample. Second, we remove accounts which have no securities with prices available at a daily level from Datastream. Third, we remove accounts with fewer than two stocks in the portfolio. [CLARIFY AT WHAT LEVEL WE MAKE THE DROPS]. Fourth, we remove accounts which we demographic data is missing. In such cases, we drop all investor $\times$ stock $\times$ days for that account from the sample. Finally, we remove accounts not showing starting position days [WHAT IS THIS?]

\ref{tab:sample_selection} reports the effects of these steps in sample selection. The table reports the number of accounts dropped due to each step in the sample restrictions, together with the number of login-days, transaction-days and buy-days dropped at each step. From the starting sample of approximately 46,000 accounts, the largest drop of accounts is due to dropping approximately 14,400 inactive accounts (31.3\% of accounts). After applying all five sample restrictions the resulting baseline sample retains 58.8\% of accounts from the unrestricted sample. Our sample restrictions tend to drop accounts with below-average logins and transactions (due to the drop of inactive accounts), hence the baseline sample retains 64.8\% of login-days and 70.9\% of sell-days. [DO WE NEED TRANSACTION DAYS AS A SEPARATE CATEGORY?, I SUGGEST DROP THIS COLUMN].

As a second step, we restrict to a sample for analysis. Two motivations drive our sample selection. First, responses to changes in left-digits is only detectable in a sample of investor $\times$ stock $\times$ days in which the left-digit changes. Moreover, we show that the response of changes in left-digit is very different depending upon when the stock is increasing in value or decreasing in value over time, in particular, selling activity occurs when prices cross left-digits from below and from above. A key element in our analysis therefore is to draw a ``price increasing sample'' and a ``price decreasing sample''. 

Second, responses to changes in left-digits are contingent upon the investor observing the change in left-digit. For example, a stock that changes left-digit over a holding period in which the investor does not make a login to the account is much less likely to be noticed compared with a change in left-digit which occurs between login days within a holding period.

We therefore define the price increasing sample and the price decreasing sample as follows. First, using the example of the price increasing sample, we identify the first day in each calendar quarter on which an investor made a login to their account.\footnote{We show later that results are unchanged when we modify the period that defines a sample to either a month, or a year, instead of a quarter.} We then define the price increasing sample as the set of login days within the quarter for which the prices on subsequent login days were always above the price on the first day and the left-digit had changed within the quarter on at least one subsequent login-day. This sample therefore provides a series of login-days through the quarter in which the price of the stock had broached a left-digit change on at least one of the login-days. 

We define the price decreasing sample using parallel sample restrictions applied to decreasing prices. Hence the price decreasing sample is defined as the set of login days within the quarter for which the prices on subsequent login days were always below the price on the first day and the left-digit had changed within the quarter on at least one subsequent login-day. Our samples are based on quarters and individual $\times$  login days during the quarter. 

\subsection{Summary Statistics}

\ref{tab:price_summary_stats_main} describes the price data for the baseline sample, price increasing sample and price decreasing sample. The unit of observation is a login-day. The baseline sample provides approximately 43.9 million login-days (the bottom row of \ref{tab:sample_selection}) Panel A summarises prices of all stocks held in the portfolio on login-days and sell-days in the first two rows, together with stocks sold in the third row. [ARE THERE 44m LOGIN DAYS, or 44M stock $\times$ LOGIN DAYS?]. The mean price of a stock in the sample of login-days is approximately \pounds8, with a median of \pounds3. 

Panels B and C summarise prices for stocks in the login-days in the price increasing sample and the price decreasing sample. Note, there are four units of left-digit in the data, pennies, tens of pennies, pounds and tens of pounds (there are only a few cases of hundreds of pounds). So, the left-digit changes of interest are pence to tens of pence, tens of pence to pounds, and pounds to tens of pounds (plus a few cases of tens of pounds to hundreds of pounds). Most stocks in the samples are prices in the range \pounds1.10 to \pounds10.10. A histogram of prices for all investor $\times$ login days is shown in \ref{fig:histogram_prices}.


