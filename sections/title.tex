% !Mode:: "TeX:US:UTF-8:Soft"

\title{\textbf{\Large{Left-Digit Bias, Investor Attention and Trading Behavior}}}

\singlespacing

\author{%
   John \\ Gathergood\thanks{University of Nottingham, School of Economics; Network for Integrated Behavioural Science. Email: \mail{john.gathergood@nottingham.ac.uk}.}%
   \and%
   George \\ Loewenstein%
   \thanks{Social and Decision Sciences, Carnegie Mellon University. Email: \mail{gl20@andrew.cmu.edu}.}%
   \and%
   Edika \\ Quispe--Torreblanca\thanks{University of Oxford, Sa\"{i}d Business School. Email: \mail{Edika.Quispe-Torreblance@sbs.ox.ac.uk}.}%
    \and%
   Neil \\ Stewart\thanks{University of Warwick, Warwick Business School. Email: \mail{Neil.Stewart@wbs.ac.uk}.}%
}

\date{%
	\vspace{1cm}\large%
	\today%
	\normalsize\\[1cm]%
}

\maketitle

\onehalfspacing

\begin{abstract}
   \noindent We show evidence of left-digit bias in the stock selling decisions of individual investors. The probability of an investor selling a stock increases discontinuously with a change in the left-digit. The probability of sale increases when stocks change left-digit due to price movements both from below and from \textit{above}. This effect is not due to limit orders, but is instead consistent with models in which salient features -- here the leftmost digit of a stock price -- attract individual attention. The degree of left-digit bias varies across investors. Left-digit bias is stronger among investors with short account tenures, lower-value portfolios and fewer stocks in their portfolios. 
   
   \vspace{2ex}\noindent%
   \textit{Keywords}: left-digit bias, investor behavior, behavioural finance

   \vspace{.5ex}\noindent%
   \textit{JEL Codes}: G40, G41, D14
\end{abstract}

\doublespacing
