\begin{econtable}[h]\small
	\caption{Sample Selection}
	\label{tab:sample_selection}
	\estauto{l c c c c c c c }{
		&\multicolumn{1}{c}{ Accounts}&\multicolumn{1}{c}{Login-Days}&\multicolumn{1}{c}{ Transaction-Days}&\multicolumn{1}{c}{Sell-Days}\\
		\midrule
		Unrestricted Sample 		&	45919	&	67734059	&	1228755	&	493041\\
\textit{Drop due to:} 		&		&		&		& \\	
\hspace{0.5cm} Inactive Accounts 		&	14370	&	7932474	&	46982	&	19562\\
\hspace{0.5cm} Unmatched Prices 		&	306	&	13009351	&	129314	&	49012\\
\hspace{0.5cm} At Least Two Stocks in Portfolio	&	3062	&	720291	&	76539	&	32652\\
\hspace{0.5cm} Missing Demographic Data 		&	1137	&	1793831	&	37427	&	16400\\
\hspace{0.5cm} Starting Position Days		&	23	&	367341	&	331557	&	25479\\
\midrule			
Baseline sample 		&	27021	&	43910771	&	606936	&	349936\\
	}
	\fignote{The unrestricted sample contains 155,300 accounts. We use a 30\% random sample of accounts.  The table detail the steps in sample selection.}
\end{econtable}

\clearpage

\begin{table}\small
	\caption{Summary Stats, Quarterly Sample}
	\label{tab:price_summary_stats_main}
	\bigskip
	\begin{adjustbox}{center}
		\estauto{lcccccccccc}<\multicolumn{9}{c}{Panel (A): Baseline Sample} \\>{
			& \multicolumn{1}{c}{N} & \multicolumn{1}{c}{Mean} & \multicolumn{1}{c}{St. Dev.} & \multicolumn{1}{c}{Min} & \multicolumn{1}{c}{Pctl(25)} & \multicolumn{1}{c}{Median} & \multicolumn{1}{c}{Pctl(75)} & \multicolumn{1}{c}{Max} \\ 	
			Price on Login Days \pounds & 87,213,924 & 7.924 & 27.054 & 0.000 & 1.150 & 3.013 & 7.510 & 15,051.630 \\ 
Price on Sell Days \pounds & 6,634,356 & 7.046 & 26.397 & 0.000 & 0.835 & 2.594 & 6.470 & 3,589.000 \\ 
Price of Stocks Sold \pounds & 701,309 & 7.282 & 30.391 & 0.000 & 0.848 & 2.656 & 6.560 & 2,062.035 \\ 
		}
	\end{adjustbox}
	
	\bigskip
	
	\begin{adjustbox}{center}
		\estauto{lcccccccccc}<\multicolumn{9}{c}{Panel (B): Price Increasing Sample} \\>{
			& \multicolumn{1}{c}{N} & \multicolumn{1}{c}{Mean} & \multicolumn{1}{c}{St. Dev.} & \multicolumn{1}{c}{Min} & \multicolumn{1}{c}{Pctl(25)} & \multicolumn{1}{c}{Median} & \multicolumn{1}{c}{Pctl(75)} & \multicolumn{1}{c}{Max} \\ 	
			All Stocks & 2,502,903 & 6.437 & 23.513 & 0.000 & 0.739 & 2.992 & 6.175 & 3,600.000 \\ 
Stocks with Prices Between \pounds 0.11 to \pounds 1.01 & 616,769 & 0.599 & 0.256 & 0.110 & 0.382 & 0.628 & 0.811 & 1.010 \\ 
Stocks with Prices Between \pounds 1.1 to \pounds 10.1 & 1,370,707 & 4.890 & 2.310 & 1.100 & 2.954 & 4.570 & 6.600 & 10.100 \\ 
Stocks with Prices Between \pounds 11 to \pounds 101 & 192,406 & 35.681 & 22.229 & 11.000 & 19.720 & 29.780 & 48.040 & 100.995 \\  
		}
	\end{adjustbox}
	
	\bigskip
	
	\begin{adjustbox}{center}
		\estauto{lcccccccccc}<\multicolumn{9}{c}{Panel (C): Price Decreasing Sample} \\>{
			& \multicolumn{1}{c}{N} & \multicolumn{1}{c}{Mean} & \multicolumn{1}{c}{St. Dev.} & \multicolumn{1}{c}{Min} & \multicolumn{1}{c}{Pctl(25)} & \multicolumn{1}{c}{Median} & \multicolumn{1}{c}{Pctl(75)} & \multicolumn{1}{c}{Max} \\ 	
			All Stocks & 5,020,091 & 4.311 & 22.496 & 0.000 & 0.180 & 1.039 & 4.515 & 3,284.000 \\ 
Between \pounds 0.10 to \pounds 1.0 & 1,372,819 & 0.514 & 0.270 & 0.100 & 0.277 & 0.488 & 0.755 & 1.000 \\ 
Between \pounds 1 to \pounds 10 & 2,195,987 & 4.508 & 2.512 & 1.000 & 2.350 & 4.115 & 6.231 & 10.000 \\ 
Between \pounds 10 to \pounds 100 & 354,766 & 25.596 & 18.917 & 10.000 & 10.890 & 20.660 & 30.235 & 99.990 \\ 
 
		}
	\end{adjustbox}
\end{table}

\clearpage

\begin{econtable}[h]\footnotesize
	\caption{Probability of Sale and Left Digit, Price Increasing Sample}
	\label{tab:regressions_increase_main}
	\estauto{l c c c c c c  }{
		& \multicolumn{5}{c}{$Probability\:  of\:  Sale_{ijt}=1$} \\ 
		%	\cmidrule(rr){2-7}
		& \multicolumn{1}{c}{(1)} & \multicolumn{1}{c}{(2)} & \multicolumn{1}{c}{(3)} & \multicolumn{1}{c}{(4)} & 
		\multicolumn{1}{c}{(5)} & \\ 
		\midrule
		\\[-2.1ex] Above Y0 = 1 (in Range Y0 to Y5) & 0.0033{***} & 0.0034{***} & 0.0029{***} & 0.0034{***} & 0.0039{***} \\ 
  & (0.0002) & (0.0002) & (0.0002) & (0.0002) & (0.0002) \\ 
  Stock Digits Y0 to Y5 &  & 0.0001 & -0.0000 & -0.0002{***} & -0.0004{***} \\ 
  &  & (0.0001) & (0.0001) & (0.0001) & (0.0001) \\ 
  Stock Digits X6 to X9 &  & -0.0002{*} & -0.0000 & 0.0000 & 0.0000 \\ 
  &  & (0.0001) & (0.0001) & (0.0001) & (0.0001) \\ 
  Constant & 0.0088{***} & 0.0086{***} & 0.0091{***} &  &  \\ 
  & (0.0002) & (0.0002) & (0.0012) &  &  \\ 
 Day FE & NO & NO & YES & YES & YES \\ 
Industry FE & NO & NO & YES & YES & YES \\ 
Account FE & NO & NO & NO & YES & YES \\ 
Stock FE & NO & NO & NO & NO & YES \\ 
Observations & \multicolumn{1}{c}{2,502,458} & \multicolumn{1}{c}{2,502,458} & \multicolumn{1}{c}{2,502,458} & \multicolumn{1}{c}{2,502,458} & \multicolumn{1}{c}{2,502,458} \\ 
R$^{2}$ & \multicolumn{1}{c}{0.0002} & \multicolumn{1}{c}{0.0002} & \multicolumn{1}{c}{0.0016} & \multicolumn{1}{c}{0.0658} & \multicolumn{1}{c}{0.0717} \\ 
 
	}
	\fignote{The unit of observation is an investor $\times$ stock $\times$ day. The samples is restricted to login days. We include only quarters in which the stocks increased in price (regarding the first observation of the quarter) and change the left most digit at least once during the quarter. Only those stocks that have changed the left most digit are included. Regressions fit an intercept for the change in the left most digit at X0 and two slopes for the left (with values in the range -3 to 0, corresponding to X6 to X9) and right (with values in the range 0 to 5, corresponding to Y0 to Y5) values. The constant shows the probability to sell the stock at when the second digit is 9 (X9). The second digit over threshold dummy shows the jump in probability when the first digit changes and so the second digit becomes 0 (X0). SE are clustered by account.}
\end{econtable}

\clearpage

\begin{econtable}[h]\footnotesize
	\caption{Probability of Sale and Left Digit, Price Decreasing Sample}
	\label{tab:regressions_decrease_main}
	\estauto{l c c c c c c  }{
		& \multicolumn{5}{c}{$Probability\:  of\:  Sale_{ijt}=1$} \\ 
		%	\cmidrule(rr){2-7}
		& \multicolumn{1}{c}{(1)} & \multicolumn{1}{c}{(2)} & \multicolumn{1}{c}{(3)} & \multicolumn{1}{c}{(4)} & 
		\multicolumn{1}{c}{(5)} & \\ 
		\midrule
		\\[-2.1ex] Above Y0 = 1 (in Range Y0 to Y5) & -0.0024{***} & -0.0038{***} & -0.0042{***} & -0.0039{***} & -0.0039{***} \\ 
  & (0.0001) & (0.0002) & (0.0002) & (0.0002) & (0.0002) \\ 
  Stock Digits Y0 to Y5 &  & 0.0002{***} & 0.0002{***} & 0.0004{***} & 0.0004{***} \\ 
  &  & (0.0000) & (0.0000) & (0.0000) & (0.0000) \\ 
  Stock Digits X6 to X9 &  & 0.0007{***} & 0.0008{***} & 0.0006{***} & 0.0005{***} \\ 
  &  & (0.0001) & (0.0001) & (0.0001) & (0.0001) \\ 
  Constant & 0.0102{***} & 0.0112{***} & 0.0147{***} &  &  \\ 
  & (0.0002) & (0.0002) & (0.0011) &  &  \\ 
 Day FE & NO & NO & YES & YES & YES \\ 
Industry FE & NO & NO & YES & YES & YES \\ 
Account FE & NO & NO & NO & YES & YES \\ 
Stock FE & NO & NO & NO & NO & YES \\ 
Observations & \multicolumn{1}{c}{5,020,091} & \multicolumn{1}{c}{5,020,091} & \multicolumn{1}{c}{5,020,091} & \multicolumn{1}{c}{5,020,091} & \multicolumn{1}{c}{5,020,091} \\ 
R$^{2}$ & \multicolumn{1}{c}{0.0002} & \multicolumn{1}{c}{0.0002} & \multicolumn{1}{c}{0.0005} & \multicolumn{1}{c}{0.0669} & \multicolumn{1}{c}{0.0710} \\ 
 
	}
	\fignote{The unit of observation is an investor $\times$ stock $\times$ day. The samples is restricted to login days. We include only quarters in which the stocks have not increased in price (regarding the first observation of the quarter) and have not changed the left most digit at least once during the quarter. Regressions fit an intercept for the change in the left most digit at X0 and two slopes for the left (with values in the range -3 to 0, corresponding to X6 to X9) and right (with values in the range 0 to 5, corresponding to Y0 to Y5) values. The constant shows the probability to sell the stock at when the second digit is 9 (X9). The second digit over threshold dummy shows the jump in probability when the first digit changes and so the second digit becomes 0 (X0). SE are clustered by account.}
\end{econtable}

\clearpage

\begin{econtable}[h]\footnotesize
	\caption{Probability of Sale and Left Digit, Splitting by Median Age}
	\label{tab:regressions_age_main}
	\estauto{l c c c c c   }{
		& \multicolumn{2}{c}{Prices Increasing Sample} &  \multicolumn{2}{c}{Prices Decreasing Sample} & \\ 
		%	\cmidrule(rr){2-7}
		& \multicolumn{1}{c}{Below Median} & \multicolumn{1}{c}{Above Median} & \multicolumn{1}{c}{Below Median} & \multicolumn{1}{c}{Above Median} &  \\ 
		\midrule
		\\[-2.1ex] Above Y0 = 1 (in Range Y0 to Y5) & 0.0075{***} & 0.0047{***} & -0.0040{***} & -0.0040{***} \\ 
  & (0.0003) & (0.0002) & (0.0002) & (0.0003) \\ 
  Stock Digits Y0 to Y5 & -0.0009{***} & -0.0006{***} & 0.0004{***} & 0.0004{***} \\ 
  & (0.0001) & (0.0001) & (0.0001) & (0.0001) \\ 
  Stock Digits X6 to X9 & -0.0003{***} & -0.0001 & 0.0007{***} & 0.0004{***} \\ 
  & (0.0001) & (0.0001) & (0.0001) & (0.0001) \\ 
 Day FE & YES & YES & YES & YES \\ 
Industry FE & YES & YES & YES & YES \\ 
Account FE & YES & YES & YES & YES \\ 
Stock FE & YES & YES & YES & YES \\ 
Observations & \multicolumn{1}{c}{2,580,896} & \multicolumn{1}{c}{2,288,818} & \multicolumn{1}{c}{2,654,464} & \multicolumn{1}{c}{2,249,414} \\ 
R$^{2}$ & \multicolumn{1}{c}{0.0866} & \multicolumn{1}{c}{0.0506} & \multicolumn{1}{c}{0.0869} & \multicolumn{1}{c}{0.0487} \\ 
 
	}
	\fignote{The unit of observation is an investor $\times$ stock $\times$ day. The samples is restricted to login days. We include only quarters in which the stocks increased/decreased in price (regarding the first observation of the quarter) and change the left most digit at least once during the quarter. Only those stocks that have changed the left most digit are included. Regressions fit an intercept for the change in the left most digit at X0 and two slopes for the left (with values in the range -3 to 0, corresponding to X6 to X9) and right (with values in the range 0 to 5, corresponding to Y0 to Y5) values. The constant shows the probability to sell the stock at when the second digit is 9 (X9). The second digit over threshold dummy shows the jump in probability when the first digit changes and so the second digit becomes 0 (Y0). SE are clustered by account.}
\end{econtable}

\clearpage

\begin{econtable}[h]\footnotesize
	\caption{Probability of Sale and Left Digit, Splitting by Gender}
	\label{tab:regressions_gender_main}
	\estauto{l c c c c c   }{
		& \multicolumn{2}{c}{Prices Increasing Sample} &  \multicolumn{2}{c}{Prices Decreasing Sample} & \\ 
		%	\cmidrule(rr){2-7}
		& \multicolumn{1}{c}{Female} & \multicolumn{1}{c}{Male} & \multicolumn{1}{c}{Female} & \multicolumn{1}{c}{Male} &  \\ 
		\midrule
		\\[-2.1ex] Above Y0 = 1 (in Range Y0 to Y5) & 0.0056{***} & 0.0059{***} & -0.0040{***} & -0.0039{***} \\ 
  & (0.0005) & (0.0003) & (0.0006) & (0.0003) \\ 
  Stock Digits Y0 to Y5 & -0.0006{***} & -0.0008{***} & 0.0004{***} & 0.0004{***} \\ 
  & (0.0001) & (0.0001) & (0.0001) & (0.0001) \\ 
  Stock Digits X6 to X9 & -0.0003 & -0.0001 & 0.0007{***} & 0.0005{***} \\ 
  & (0.0002) & (0.0001) & (0.0002) & (0.0001) \\ 
 Day FE & YES & YES & YES & YES \\ 
Industry FE & YES & YES & YES & YES \\ 
Account FE & YES & YES & YES & YES \\ 
Stock FE & YES & YES & YES & YES \\ 
Observations & \multicolumn{1}{c}{429,057} & \multicolumn{1}{c}{2,073,846} & \multicolumn{1}{c}{401,271} & \multicolumn{1}{c}{2,127,011} \\ 
R$^{2}$ & \multicolumn{1}{c}{0.0731} & \multicolumn{1}{c}{0.0730} & \multicolumn{1}{c}{0.0774} & \multicolumn{1}{c}{0.0749} \\ 
 
	}
	\fignote{The unit of observation is an investor $\times$ stock $\times$ day. The samples is restricted to login days. We include only quarters in which the stocks increased/decreased in price (regarding the first observation of the quarter) and change the left most digit at least once during the quarter. Only those stocks that have changed the left most digit are included. Regressions fit an intercept for the change in the left most digit at X0 and two slopes for the left (with values in the range -3 to 0, corresponding to X6 to X9) and right (with values in the range 0 to 5, corresponding to Y0 to Y5) values. The constant shows the probability to sell the stock at when the second digit is 9 (X9). The second digit over threshold dummy shows the jump in probability when the first digit changes and so the second digit becomes 0 (Y0). SE are clustered by account.}
\end{econtable}

\clearpage

\begin{econtable}[h]\footnotesize
	\caption{Probability of Sale and Left Digit, Splitting by Portfolio Value}
	\label{tab:regressions_portvalue_main}
	\estauto{l c c c c c   }{
		& \multicolumn{2}{c}{Prices Increasing Sample} &  \multicolumn{2}{c}{Prices Decreasing Sample} & \\ 
		%	\cmidrule(rr){2-7}
		& \multicolumn{1}{c}{Below Median} & \multicolumn{1}{c}{Above Median} & \multicolumn{1}{c}{Below Median} & \multicolumn{1}{c}{Above Median} &  \\ 
		\midrule
		 Above Y0 = 1 (in Range Y0 to Y5) & 0.0153{***} & 0.0081{***} & -0.0082{***} & -0.0038{***} \\ 
  & (0.0013) & (0.0011) & (0.0009) & (0.0007) \\ 
  Stock Digits Y0 to Y5 & -0.0016{***} & -0.0008{***} & 0.0009{***} & 0.0005{***} \\ 
  & (0.0003) & (0.0003) & (0.0002) & (0.0002) \\ 
  Stock Digits X6 to X9 & 0.0001 & -0.0005 & 0.0010{**} & 0.0001 \\ 
  & (0.0005) & (0.0004) & (0.0004) & (0.0003) \\ 
 Day FE & YES & YES & YES & YES \\ 
Industry FE & YES & YES & YES & YES \\ 
Account FE & YES & YES & YES & YES \\ 
Stock FE & YES & YES & YES & YES \\ 
Observations & \multicolumn{1}{c}{171,087} & \multicolumn{1}{c}{145,155} & \multicolumn{1}{c}{248,029} & \multicolumn{1}{c}{192,776} \\ 
R$^{2}$ & \multicolumn{1}{c}{0.1701} & \multicolumn{1}{c}{0.0886} & \multicolumn{1}{c}{0.1445} & \multicolumn{1}{c}{0.0750} \\ 
 
	}
	\fignote{The unit of observation is an investor $\times$ stock $\times$ day. The samples is restricted to login days. We include only quarters in which the stocks increased/decreased in price (regarding the first observation of the quarter) and change the left most digit at least once during the quarter. Only those stocks that have changed the left most digit are included. Regressions fit an intercept for the change in the left most digit at X0 and two slopes for the left (with values in the range -3 to 0, corresponding to X6 to X9) and right (with values in the range 0 to 5, corresponding to Y0 to Y5) values. The constant shows the probability to sell the stock at when the second digit is 9 (X9). The second digit over threshold dummy shows the jump in probability when the first digit changes and so the second digit becomes 0 (Y0). SE are clustered by account.}
\end{econtable}

\clearpage

\begin{econtable}[h]\footnotesize
	\caption{Probability of Sale and Left Digit, Splitting by Account Tenure}
	\label{tab:regressions_tenure_main}
	\estauto{l c c c c c   }{
		& \multicolumn{2}{c}{Prices Increasing Sample} &  \multicolumn{2}{c}{Prices Decreasing Sample} & \\ 
		%	\cmidrule(rr){2-7}
		& \multicolumn{1}{c}{Below Median} & \multicolumn{1}{c}{Above Median} & \multicolumn{1}{c}{Below Median} & \multicolumn{1}{c}{Above Median} &  \\ 
		\midrule
		\\[-2.1ex] Above Y0 = 1 (in Range Y0 to Y5) & 0.0045{***} & 0.0034{***} & -0.0027{***} & -0.0023{***} \\ 
  & (0.0004) & (0.0003) & (0.0003) & (0.0003) \\ 
  Stock Digits Y0 to Y5 & -0.0004{***} & -0.0003{***} & 0.0005{***} & 0.0003{***} \\ 
  & (0.0001) & (0.0001) & (0.0001) & (0.0001) \\ 
  Stock Digits X6 to X9 & 0.0002 & -0.0001 & 0.0004{**} & 0.0003{**} \\ 
  & (0.0002) & (0.0001) & (0.0001) & (0.0001) \\ 
 Day FE & YES & YES & YES & YES \\ 
Industry FE & YES & YES & YES & YES \\ 
Account FE & YES & YES & YES & YES \\ 
Stock FE & YES & YES & YES & YES \\ 
Observations & \multicolumn{1}{c}{1,235,031} & \multicolumn{1}{c}{1,267,427} & \multicolumn{1}{c}{1,280,135} & \multicolumn{1}{c}{1,247,863} \\ 
R$^{2}$ & \multicolumn{1}{c}{0.0825} & \multicolumn{1}{c}{0.0608} & \multicolumn{1}{c}{0.0825} & \multicolumn{1}{c}{0.0673} \\ 
 
	}
	\fignote{The unit of observation is an investor $\times$ stock $\times$ day. The samples is restricted to login days. We include only quarters in which the stocks increased/decreased in price (regarding the first observation of the quarter) and change the left most digit at least once during the quarter. Only those stocks that have changed the left most digit are included. Regressions fit an intercept for the change in the left most digit at X0 and two slopes for the left (with values in the range -3 to 0, corresponding to X6 to X9) and right (with values in the range 0 to 5, corresponding to Y0 to Y5) values. The constant shows the probability to sell the stock at when the second digit is 9 (X9). The second digit over threshold dummy shows the jump in probability when the first digit changes and so the second digit becomes 0 (Y0). SE are clustered by account.}
\end{econtable}

\clearpage

\begin{econtable}[h]\footnotesize
	\caption{Probability of Sale and Left Digit, Splitting by Number of Stocks}
	\label{tab:regressions_numstocks_main}
	\estauto{l c c c c c   }{
		& \multicolumn{2}{c}{Prices Increasing Sample} &  \multicolumn{2}{c}{Prices Decreasing Sample} & \\ 
		%	\cmidrule(rr){2-7}
		& \multicolumn{1}{c}{Below Median} & \multicolumn{1}{c}{Above Median} & \multicolumn{1}{c}{Below Median} & \multicolumn{1}{c}{Above Median} &  \\ 
		\midrule
		\\[-2.1ex] Above Y0 = 1 (in Y0 - Y5) & 0.0088{***} & 0.0031{***} & -0.0049{***} & -0.0030{***} \\ 
  & (0.0003) & (0.0002) & (0.0003) & (0.0002) \\ 
  Digits Y0 - Y5 & -0.0011{***} & -0.0004{***} & 0.0005{***} & 0.0004{***} \\ 
  & (0.0001) & (0.0001) & (0.0001) & (0.0000) \\ 
  Digits X6 - X9 & -0.0003{***} & -0.0001 & 0.0009{***} & 0.0002{**} \\ 
  & (0.0001) & (0.0001) & (0.0001) & (0.0001) \\ 
 Day FE & YES & YES & YES & YES \\ 
Industry FE & YES & YES & YES & YES \\ 
Account FE & YES & YES & YES & YES \\ 
Stock FE & YES & YES & YES & YES \\ 
Observations & \multicolumn{1}{c}{2,650,726} & \multicolumn{1}{c}{2,218,988} & \multicolumn{1}{c}{2,492,742} & \multicolumn{1}{c}{2,411,136} \\ 
R$^{2}$ & \multicolumn{1}{c}{0.0908} & \multicolumn{1}{c}{0.0377} & \multicolumn{1}{c}{0.0934} & \multicolumn{1}{c}{0.0329} \\ 
 
	}
	\fignote{The unit of observation is an investor $\times$ stock $\times$ day. The samples is restricted to login days. We include only quarters in which the stocks increased/decreased in price (regarding the first observation of the quarter) and change the left most digit at least once during the quarter. Only those stocks that have changed the left most digit are included. Regressions fit an intercept for the change in the left most digit at X0 and two slopes for the left (with values in the range -3 to 0, corresponding to X6 to X9) and right (with values in the range 0 to 5, corresponding to Y0 to Y5) values. The constant shows the probability to sell the stock at when the second digit is 9 (X9). The second digit over threshold dummy shows the jump in probability when the first digit changes and so the second digit becomes 0 (Y0). SE are clustered by account.}
\end{econtable}