\section{Results}

\subsection{Main Results}

Our main result is shown in \ref{fig:left_digit_sell_main}, which shows the probability of selling a stock by the leftmost price digit of the stock. The left-side plots in panels A and B take the samples of observations from the price increasing sample and price decreasing sample and stack observations (investor $\times$ stock $\times$ login days) by the leftmost two digits of the observation, centred on the change in left digit between some integer $X$ and the next $Y = X + 1$. For example, an investor holds a stock and, in the price increasing sample, sees prices on login-days of 18p, 22p and 26p. These observations stack onto the x-axis at $X8$, $Y2$ and $Y6$, centring on the change in left-digit from 1 to 2 with $X = 1 , Y = 2$. Another investor holds a stock, in the price increasing sample, sees prices on login days of 361p, 389p and 430p. These observations stack onto the x-axis at $X6$, $X8$ and $Y3$, centring on the change in left-digit from 3 to 4, with $X = 3 , Y = 4$.\footnote{Stocks that cross multiple left-digits in the quarter contribute observations centred around each left-digit change.} One stock $\times$ day can contribute to multiple observations, if more than one investor makes a login to their account on the day. By way of contrast, a stock $\times$ day might contribute no observations if no investor makes a login to their account on the day.

The first clear pattern in the left-side plots is that the probability of sale jumps upwards when the stock crosses a left-digit from below in the price increasing sample, and the probability of sale jumps downwards when the stock crosses a left-digit from above in the price-decreasing sample. In the price increasing sample, the increase is from approximately 0.008 to 0.0014, a 75\% increase, whereas in the price decreasing sample, the increase is from approximately 0.008 to 0.0012, a 50\% increase.\footnote{The probability of sale is higher in the price increasing sample compared with the price decreasing sample, indicating that investors are more likely to sell stocks when prices increase compared with when prices decrease.} 

A second clear pattern in the left-side plot is that the probability of sale jumps upwards when the price crosses the left-digit, and remains elevated for the next adjoining units, in the price increasing sample $Y1, Y2, Y3, ...$, in the price decreasing sample $X9, X8, X7,...$. One explanation for this pattern is that the investor's response to a change in left-digit does not necessarily occur when the price crosses $Y0$, as the investor may not observe the stock at price $Y0$ if they do not make a login at that point in time. Instead, the investor may only observe the change in left-digit at the point at which they log in, and by that point in time the stock have have reached $Y1, Y2, Y3,...$ in the price increasing sample, for example.

The right-side plots in each panel illustrate the probability of sale across the full range of leftmost two digits of prices in the sample, unstacking the data from the left-side plot by leftmost two digits. In the example above, the series of prices 18p, 22p and 26p enter into the x-axis bins for 18, 20 and 26; while the series of prices 361p, 389p and 430p enter into the ex-axis bins for 36, 38 and 43. In both panels, each bin in which the right-digit of the leftmost two digits ends 9 is colored blue, with each bin in which the right-digit of the leftmost two digits ends 0 is colored red.

The clear pattern in the right-side plots is that the probability of sale jumps when the price crosses a leftmost digit from below in Panel A, and from above in Panel B. This is seen across the broad range of leftmost two digits of prices. The plot also makes clear the ``ladder'' pattern whereby the elevated probability of sale at each crossing of the leftmost digit steps down as the price moves further upwards (in Panel A) or downwards (in Panel B), with the probability jumping again at the next change in leftmost digit.

This pattern in the probability of sale when a stock price crosses a leftmost digit is \textit{not} seen when one pools all observations. Pooling all observations of login-days from the baseline sample (approximately 84 million observations). \ref{fig:left_digit_sell_all} reproduces the figures from \ref{fig:left_digit_sell_main}. The left-side plot shows no discontinuity in the probability of sale when price crosses the leftmost digit. In the figure, the probability of sale is slightly higher at lower prices, but the difference in probability across the x-axis very small (in the range 0.0079 to 0.0085), compared with the jump in probability at the crossing of the left-digit of 0.006 in the price increasing sample and 0.004 in the price decreasing sample. The right-side plot also shows no difference in the probability of sale between the red bars and blue bars.

We estimate the size of the left-digit effect in \ref{tab:regressions_increase_main} and \ref{tab:regressions_decrease_main}, which show regression estimates for the price increasing sample and price decreasing samples respectively. The dependent variable in each regression is a dummy variable for whether the observation is a sale (either partial or total sale). The specification in Column 1 includes only a dummy for whether the stock price is $Y0$ or above. The coefficient on the dummy variable is 0.0044 in the price increasing sample. This implies an increase in the probability of sale when the stock price crosses the left-digit from below, evaluated against the constant term, of 52\%. The coefficient on the dummy variable is -0.0024 in the price decreasing sample. This implies an increase in the probability of sale when the stock price crosses the left-digit from above, evaluated against the constant term, of 24\%. 

Additional columns in both tables add further controls to the econometric specification. The specification in Column 2 of each table adds slope terms for the range $Y0 - Y5$ and $X6 - X9$. Subsequent Columns add a series of fixed effects: day fixed effects, industry fixed effects (using industry classifications based on Datastream Industry Classification Benchmark (ICB)), account fixed effects and finally stock fixed effects. The specification in Column 5 therefore exploits within-investor, within-stock variation in the probability of sale, conditioning on day differences in the likelihood of sale. In this richest specification, the coefficients on the dummy variable of 0.0061 and -0.0040 imply an increase in the probability of sale when the stock price crosses the left-digit from below of 71\% and an increase in the probability of sale when the stock prices crosses the left-digit from above of 40\%.\footnote{These percentage increases in the probability of sale are calculated from the coefficient on the constant term in Column 1 of each table.}

The main results in  \ref{fig:left_digit_sell_main} pool over leftmost digits in pence, pounds and tens of pounds.\footnote{There are very few observations of stock prices in which the leftmost digit is in hundreds of pounds.}. In further analysis, we reproduce the figures show in  \ref{fig:left_digit_sell_main} and regression estimates in \ref{tab:regressions_increase_main} and \ref{tab:regressions_decrease_main} for separate samples in which stock prices are in pence, pounds and tens of pounds. \ref{fig:left_digit_sell_increase_main} and \ref{fig:left_digit_sell_decrease_main} each show three panels with the sample split into observations with prices in the range \pounds0.11 to \pounds1.01 in Panel A, \pounds1.01 to \pounds10.1 in Panel B, and \pounds11 to \pounds101 in Panel C of each figure. The number of observations is uneven distributed across these subsample, with approximately 50\% of observations from the main samples in Panel B of each figure and only 7\% of observations from the main sample in Panel C of each figure.  The jump in probability of sale when the price crosses the leftmost digit from below, or when the price crosses the leftmost digit from above, is seen in each sub-sample.\footnote{The jump in probability of sale when the leftmost digit crosses zero is less clear in Panel C of each plot, which contains the highest priced stocks in each sample, and contributed the smallest percentage of observations in each main sample.} \ref{tab:reg_subsamples_increase} and \ref{tab:reg_subsamples_decrease} report regressions for the subsamples by pennies, pounds and tens of pounds. Estimates from these models show very similar results to the main regression estimates.

\subsection{Robustness Tests}

\subsubsection{Limit Orders}

Why do we observe the jump in probability of sale when the price crosses a left-digit? One potential explanation for the pattern seen in the price increasing sample is that investors set limit orders with strike prices at round numbers (e.g., base-10 or base-100 numbers). Two observations of the patterns seen in \ref{fig:left_digit_sell_main} suggest this is unlikely to explain the pattern we observe in full. First, while the use of limit orders might contribute to the pattern observed in Panel A, it would not explain the pattern observed in Panel B. Second, in liquid markets we would expect limit order to be executed at the strike price which, if a round number, would cause a spike in the probability of sale only at $Y0$ in the illustrations, not an elevated probability of sale at $Y1, Y2, Y3, ...$. There are, however, two counter-arguments to this. First, if individuals place limit order outside of trading hours, the price may have risen further above Y0 by the time the brokerage executes the order (overnight orders form a queue on the broker's order book, and orders appearing later in the queue may execute at a price further away from the strike price). Second, if the stock is illiquid the brokerage may only be able to execute the order once the price has risen further above Y0 (again, due to queueing). 

As a first set of robustness tests, we therefore exclude observations of sales which have a higher probability of being placed as a limit order. For example, the share of limit orders in the sample of orders placed out-of-hours is likely to be higher than the share of limit orders in the sample of orders placed during market hours. We exclude two sets of observations. First, observations of sales where the order is completed by the broker outside UK market hours (8am to 4.30pm). Second, observations of sales where the investor may a login to their account on the previous day and may have placed a sell order using a limit order\footnote{Likewise, we also exclude observations of sales executed on Mondays when  the investor log in during the weekend}. Panels A and B of \ref{fig:limit_order_figures} show the impact of implementing these exclusions. The pattern seen in the main analysis, of a jump in the probability of sale when the price crosses the leftmost digit from below in the price increasing sample, is seen in both panels.

Second, we restrict the sample to only the most liquid stocks, focusing on stocks in the FTSE100 only. In this sample, the likelihood of round-number limit orders executing at their strike price is higher compared with less liquid stocks in the FTSE250 or FTSE All-Share indices. If limit orders fulfilling at round number strike prices are responsible for the jump in probability of sale when the leftmost digit crosses zero from below, we should therefore see a sharp ``spike'' in sales at $Y0$ in the sample. Panel C, which restricts to FTSE100 stocks only, shows an elevated probability of sales at prices $Y1, Y2, Y3, ...$ in the same way as seen in the main results. This again suggests that the pattern we see in the main results is not explained by investors using limit orders.

Third, we follow an approach to identifying limit order trades suggested by  Linnainmaa (2010). Linnaimaa (2010) explores the role of limit orders as a potential explanation for the existence of the disposition effect, arguing that the disposition effect (among other features of individual investor trading behavior) can be explained by investors' use of limit orders. The paper proposes a method for identifying trades more likely to have been executed via limit orders.  

The approach proceeds as follows. We first take all observations of buys and sell events for each investor in the price increasing sample. We then regress a buy-versus-sell indicator (a dependent variable that takes the value of one when an investor sells a stock and the value of zero when an investor purchases a stock) against the daily return of an stock, for each investor. Following Linnaimaa, we included investors with at least 10 trades. According to Linnaimaa's method, the same-day return coefficient is significantly positive for limit-order trades, but significantly negative for market-order trades (because individuals who are net buyers when the stock price falls, and net sellers when the stock price rises, are likely to be limit-order traders; while individuals who submit market orders often trade in the direction of the same-day return, and hence against limit order traders). Using this method, we are able to exclude from the sample investors for whom a positive coefficient is estimated from the analysis (\textcolor{blue}{839 investors corresponding to 316,132 observations and 6,144 sell events, 11\% of the total number of account $\times$ stock $\times$ sell days}).
\footnote{\textcolor{blue}{Using trading records of all investors in Finland for the period 1995-2002, Linnaimaa shows that the fraction of actual limit-order trades in the sample of investors with positive coefficient, following this approach, is 65\%}}

We show the effect of excluding limit order investors on our main results in Panel D of \ref{fig:limit_order_figures}. The pattern seen is the main sample is unchanged in this further restricted sample, with a clear increase in the probability of sale when the price increases through a change in the leftmost digit. We show regression estimates using each of the samples in \ref{fig:limit_order_figures} in \ref{tab:limit_order_2} and  \ref{tab:limit_order_2}. The regression results show the same patterns as those using the main sample.

\subsection{Simulation}

The main results are shown using samples of stocks which increase, or decrease, in price, crossing a left-digit within the period of observations (a calendar quarter). This sample selection therefore contains a large proportion of observations with prices at or just above $Y0$ in the price increasing sample, or at or just below $Y0$ in the price decreasing sample, as this is the minimum criteria for inclusion in the price increasing sample, or price decreasing sample, respectively. Hence, a large share of the total number of observations of sales in each sample is clustered around $Y0$.

This sample selection should not mechanistically create a higher \textit{probability} of selling a stock at $Y0$ compared with other leftmost two digit prices in the range $X6 - Y5$ used in the analysis. However, to test for this we conduct a simulation analysis in which we assign sales randomly to investor $\times$ stock $\times$ days based on the average probability to sell observed in the data. If our main results are due to sample selection, we should see the jumps in probability of sale when sells are randomly assigned across observations. \ref{fig:sample_selection_test} shows that with randomly allocated sales we see no evidence of discontinuity in the probability of sale when the leftmost digit changes. This suggests that our main result does not arise mechanistically due to the sample selection.

\subsection{Sensitivity Tests}

\subsubsection{Variation in Time Period}

We test the sensitivity of our main estimates to the time period over which changes in the leftmost digit are calculated in the price increasing sample and the price decreasing sample. In our main analysis this time period is a calendar quarter, with observations restricted to the set of login days within the quarter for which the prices on subsequent days were always above the price on the first day and the left-digit had changed within the quarter at least once on a subsequent login-day. 

We show the sensitivity of our results to shortening the time period for this sample restrict to either one month, or extending the sample restriction to one year in \ref{fig:left_digit_sell_monthly} and as a year in \ref{fig:left_digit_sell_annual}. The patterns seen in these figures are very similar to those seen in the main analysis. \ref{tab:summary_stats_annual_monthly} reports for summary statistics for the stock prices in the monthly and yearly samples, with \ref{tab:price_increasing_monthly_annual} and \ref{tab:price_decreasing_monthly_annual} showing regression estimates.

\subsubsection{Sell-Day Sample}

We also test the sensitivity of the main results to restricting the sample to sell-days instead of login-days. One could conduct the analysis of all days, login-days, sell-days or some other restriction to types of days. In our main analysis we use login-days as these are days on which the probability of an investor paying attention to the prices of stocks in their portfolio is higher, because they make a login to the account. We further restrict to sell-days as on sell-days we might be expect an even higher probability that the investor pays attention to the prices of stocks in the portfolio, given that they make a sale.

We show results when restricting the baseline sample to sell-days in  \ref{fig:left_digit_sell}. The same pattern is seen as in the main analysis, of a jump in the probability of sale when the price crosses the leftmost digit from below in the price increasing sample, and a jump in the probability of sale when the prices crosses the leftmost digit from above in the price decreasing sample. Regression estimates are shown in \ref{tab:regressions_sellsample_increase} and \ref{tab:regressions_sellsample_decrease}. We also show results from this analysis in sub-samples by pennies, pounds and tens of pounds in \ref{fig:left_digit_sell_increase_sellsample} and \ref{fig:left_digit_sell_decrease_sellsample}, with regression analysis for these sub-samples shown in \ref{tab:regression_sellsubsamples_increase} and \ref{tab:regression_sellsubsample_decrease}.

\section{Who Exhibits Left-Digit Bias?} 

In this section we explore heterogeneity in left-digit bias across investor characteristics and investor portfolio characteristics. First, we consider heterogeneity by investor gender and age. Previous studies show gender and age differences in trading behavior. To investigate, we split the sample by investor gender and also, separately, by investor age (splitting the sample at the age of the median investor). We then estimate our main models on both samples separately, for both the price increasing sample and the price decreasing sample. This approach allows the coefficients on all variables (not just the left-digit change dummy) to vary across samples.

Results for investor sub-samples by age and gender are shown in \ref{tab:regressions_age_main} and \ref{tab:regressions_gender_main}. Results in \ref{tab:regressions_age_main} reveal a stronger left-digit effect among younger investors compared with older investors in the price increasing sample, with no difference by age group in the price decreasing sample. While \ref{tab:regressions_gender_main} indicates no differences in left-digit effects by gender in either the price increasing sample or the price decreasing sample. Hence, there appears to be little variation in lef-digit sensitivity across investors by age and gender.

Results for sub-sample by portfolio characteristics reveal a higher degree of heterogeneity. \ref{tab:regressions_portvalue_main} - \ref{tab:regressions_numstocks_main} show results by sub-samples according to median splits of portfolio value, account tenure and number of stocks held in the portfolio. Results in \ref{tab:regressions_portvalue_main} show that the left-digit coefficient is larger among lower-value accounts. The coefficient on the left-digit dummy is approximately three times as large for investors with below-median portfolio values in both the price increasing sample and in the price decreasing sample. 




