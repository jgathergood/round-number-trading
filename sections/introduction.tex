\section{Introduction} \label{sec:introduction}

A large literature in finance focusing on the disposition effect \citep [e.g.,][]{weber1998disposition, frazzini2006disposition, barberis2009drives, ben2012investors,  li2013prospect} -- the distaste for selling stocks at a nominal loss -- addresses the question of when people \textit{don't} like to sell stocks.  However, beyond this strong regularity, there is very little research focusing on when, exactly, investors \textit{do} sell stocks. Are there specific events that trigger or suppress the sale of a stock? 
Recent research, which finds that the buy decisions of professional traders are quite sensible -- the stocks they buy are more likely to rise in value than those they don't buy -- but that their sell decisions are worse than random, further highlights the need for a better understanding of when stock sales occur \citep{akepanidtaworn2019selling}. 

In this paper we investigate how individual investors respond to changes in stock prices, in particular, changes in the left-digit of a stock price. While movements in prices associated with a change in left-digits -- e.g., between \$5.99 and \$6.01 -- are materially identical to same-magnitude movements in prices that do not change left-digits -- e.g. between \$5.76 and \$5.78 -- we find large effects of changes in left-digits on the probability of an investor selling a stock:  Investors are significantly -- 70\% -- more likely to sell stocks when their price crosses a round-number price threshold from below. By the same token, we find that investors are also more likely -- 40\% -- to sell stocks immediately after they cross a round number threshold from \textit{above}. Hence movements in prices that change left-digits, both from below and from above, lead to increases in the probability that a stock is sold by retail investors.  

We document these interrelated patterns using a data set of transactions made by online retail investors. We draw from our sample of investor trading records samples in which investors hold stocks through periods of rising prices, specifically episodes in which a stock price rises through left-digit changes and, separately, samples in which investors hold stocks through periods of falling prices, in which a stock price falls through left-digit changes. We construct these samples such that we know the investor had a high probability of seeing the change in left-digit, doing so by defining our samples using prices matched to login events. Importantly, we only observe an effect of left-digit changes on investor behaviour in samples defined by login events, implying that seeing the left-digit change is the driver of the subsequent increased probability of sale. We demonstrate the robustness of our findings across different empirical inspections, and rule out limit orders as an alternative explanation. We also show that the the left-digit effect is stronger among less experienced investors and those with smaller portfolios (fewer stocks and lower portfolio values).  

We interpret the large increase in probability of sale when a stock price crosses a left-digit as arising due to the salience of left-digits in displayed prices. A change in the left-digit is more salient compared with the following digits, and, as a more salient feature of the price, is more likely to attract investor attention (as in the models of Bordalo et al.,  \citeyear{bordalo2012salience, bordalo2013salience}, \citealp{ kHoszegi2013model}, and \citealp{bushong2015model}). People naturally focus attention on attributes that are more salient, as has been shown using responses to sales taxes (\citealp{chetty2009salience}; \citealp{finkelstein2009ztax}). Previous studies also show that the salience of other aspects of stock prices within an investors portfolio matters for trading decisions. For example \cite{hartzmark2015} shows that individuals are more likely to sell the extreme winning and extreme losing positions in their portfolio, a form of rank effect. 

Our study uses individual-level investor trading records provided by Barclays Stockbroking, an online execution-only brokerage service operating in the United Kingdom. The data provided covers a four-year period and includes records of portfolio holding and trades at the daily level. The data also provide daily dummy variables for whether the investor made a login to the account. These allow us to measure prices observed by the investor with high probability - prices on days on which the investor made a login to the account - compared to prices on days on which the investor did not make a login to the account, when they would be less likely to observe a price shifting over a round number threshold.

Our research design focuses on holding periods in which the investor observed a change in the left-digit of the stock. We select samples of holding periods over quarters in which the investor held a stock which began the quarter below (above) a left-digit change, and then increased (decreased) through a left-digit change.\footnote{Our results are unchanged when using a calendar month, or year, as the holding period in the sample selection.} This sample selection allows us to restrict to observed changes in left-digits. In this sample, we observe large effects of changes in left-digits.

Our main estimates are robust to tests to account for limit orders. An increased probability of sale when a stock crosses a left-digit could arise, not due to individuals responding to a change in the left-digit, but instead due to individuals placing a limit order with a strike price at the change in left-digit. However, we rule out limit orders as the mechanism driving our results using a variety of tests, including the method suggested by \cite{linnainmaa2010limit} to identify and exclude limit order trades from the data. 

In further analysis, we examine heterogeneity in the left-digit effect on stock sales across investor characteristics and portfolio characteristics. We find little heterogeneity in responses to left-digit changes across investors by age and gender, but large differences across investors by portfolio value, number of stocks held in the portfolio and account tenure. Investors with short tenures, few stocks and smaller portfolios exhibit a stronger response to changes in the left-digit of the stock price, both from below and from above.

In addition to providing new insights into investor sell decisions, our paper contributes to the diverse literature on left-digit bias, which is the tendency to pay more attention, and give greater weight in decision making, to the leftmost digit of a number relative to other digits (\citealp{poltrock1984comparative}).\footnote{Other forms of bias in processing number values have also been shown in the literature, including the tendency of individuals to process small numbers on a linear scale while processing large numbers on a logarithmic scale (\citealp{roger2018behavioral}) and to exhibit exponential growth bias (\citealp{stango2009exponential}).} 

Several papers have documented economic consequences of the left digit bias. \cite{lacetera2012heuristic}, for example, find left-digit bias in the processing of odometer values, leading to discontinuous drops in sale prices at 10,000-mile odometer thresholds. \cite{shlain2018more} structurally estimates the magnitude of left-digit bias using retail pricing data, finding that consumers respond to a 1-cent increase from a 99-ending price as if it were a 15-25 cent increase.\footnote{Relatedly, laboratory studies have found that prices ending in a nine unit are perceived to be disproportionately smaller than prices ending in the following zero unit, e.g., 99 cents compared with \$1 (\citealp{thomas2005penny}; \citealp{manning2009price}).} And research on physician decision making \citep{olenski2020behavioral} find that patients hospitalized with acute myocardial infarction 2 weeks after, as compared with 2 weeks before, their 80th birthday are significantly less likely to undergo coronary-artery bypass graft surgery.  

Other strands in the literature on left-digit bias examine how round numbers act as reference points. For example, \cite{allen2016reference} show that round numbers act a reference points for marathon finishing times. \cite{pope2011round} show how round numbers act as goals in a variety of settings, such as professional baseball players who modify their behavior as the season is about to end to finish with a batting average just above rather than below .300, and high school students, who are more likely to retake the SAT after obtaining a score just below rather than above a round number. \cite{pope2015focal} show that heaping of agreed house sales prices at \$50,000 units can be explained by round numbers acting as focal points in sale negotiations. \cite{bhattacharya2012penny} find that stock traders focus on round numbers as cognitive reference points for value, evidenced by excess buying (selling) by liquidity demanders at all price points one penny below (above) round numbers.\footnote{A further strand in the literature focuses on whether some left-digit values are more prominent in individuals choices compared with others. \cite{albers1983prominence} presents a theory of the prominence of numbers in the decimal system in which a subset of round numbers are particularly prominent and hence more likely to be chosen. We find evidence in our credit card payments data consistent with the idea that some round numbers are more prominent than others.}

Our study also contributes to the broader behavioral literature on individual investor behaviour. Investors typically hold only a few stocks and exhibit biases such as over-trading \citep{barber2000}, sensitivity to gains compared with losses \citep{odean1998} and rank effects \citep{hartzmark2015}. For review of the behavioral finance literature see \cite{hirshleifer2015} and \cite{barberis2018}, and the review of investor behavior of \cite{barber2013}.

The selling behavior of individual investors around left-digit changes in stock prices in our sample also echoes previous studies of the behaviour of stock prices around left-digit thresholds. Using the Trade and Quote data set from the New York Stock Exchange, \cite{bhattacharya2012penny} study the behaviour of bid and ask orders placed by securities traders, finding excess buying by liquidity demanders at all price points one penny below round numbers, and excess selling by liquidity suppliers at one penny above round numbers. These patterns are partially due to limit order clustering, as shown by \cite{bourghelle2009limit} and \cite{chiao2009price}. There is evidence that individual investors who place limit orders at round numbers exhibit lower cognitive ability and worse portfolio performance \citep{kuo2015cognitive}. In contrast to these studies, we focus on left-digit bias as arising from an immediate response when investors observe prices crossing left-digits rather than as a mechanical response due to the preference to set limit orders at round numbers.  

The paper proceeds as follows. \ref{sec:data} describes the data we use in the analysis, the steps in sample selection and presents summary statistics. \ref{sec:results_main} presents the main results, together with robustness tests and sensitivity checks. \ref{sec:results_heterogeneity} presents results on the heterogeneity in left--digit effect across investor characteristics and portfolio characteristics. 


