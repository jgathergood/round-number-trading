Although there is by now a large literature in finance addressing the question of when people \textit{don't} like to sell stocks -- specifically focusing on the disposition effect, the distaste for selling stocks at a nominal loss -- beyond this strong regularity there is very little research focusing on when, exactly, investors \textit{do} sell stocks. Are there specific events that trigger the sale of a stock? 
Recent research \citep{akepanidtaworn2019selling}, which finds that the buy decisions of professional traders are quite sensible -- the stocks they buy are more likely to rise in value than those they don't buy -- but that their sell decisions are worse than random, further highlights the need for a better understanding of when stock sales occur. 

While not providing a comprehensive theory, nor a broad empirical investigation, of when people sell stocks, in this paper we address one event that, we predicted and found, has a substantial effect on sales: People are significantly -- xx\% -- more likely to sell stocks when their price crosses a round-number price threshold from below -- e.g., rising from below \$30 per share to above \$30 per share. By the same token, we find that investors are  less likely to sell stocks immediately after they cross a round number threshold from above.  We document these interrelated patterns using a data set of transactions made by online retail investors, demonstrate its robustness across different empirical inspections, and rule out limit orders as an alternative explanation. 

Left-digit bias is the tendency to focus on the leftmost digit of a number while paying less attention to other digits (\citealp{poltrock1984comparative}). Prior research on the left-digit bias has shown automobiles depreciate disproportionately when their milage crosses a around number threshold.  Research on physician decision making likewise find that patients hospitalized with acute myocardial infarction 2 weeks after, as compared with 2 weeks before, their 80th birthday were significantly less likely to undergo coronary-artery bypass graft surgery.  And research \citep{shlain2018more} shows not only that 99 cent pricing works -- that consumers respond to a one cent increase of \$.99 to \$1.00 as if it was a 15-25 cent difference, but also that firms exploit this bias less than they would if they were maximizing profits.  Our contribution is to show that the left-digit bias strongly affects the behavior of investors.


